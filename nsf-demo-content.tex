
% Compact lists as much as possible, make lists inside paragraphs easier.
\usepackage[inline]{enumitem}
\setlist{nosep}

% Tables
\usepackage{booktabs}
\setlength{\tabcolsep}{0.3em} % More compact tables by default
\usepackage{array}
\newcolumntype{P}[1]{>{\raggedright\arraybackslash}p{#1}} % Allow ragged right paragraph columns
\usepackage{multicol}

% Units (for science drivers and performance measurements)
\usepackage[group-separator={,},
  binary-units=true,
  per-mode=symbol,
  detect-all=true]{siunitx}
\newcommand{\dol}[2][]{\SI{#2}[\$]{#1}}
% use as \dol[\mega\relax]{1} for $1M, \dol{123} for $123.

% Graphics
\usepackage{graphicx}

% Better cross-references
\usepackage{cleveref}

\usepackage[sorting=none,sortcites=true,citestyle=numeric-comp]{biblatex}
\usepackage{pifont,xcolor} % for project needs table
\setlength\bibitemsep{\baselineskip}
\addbibresource{nsf-demo.bib}
\begin{document}
\chapter*{Project Summary} % 1 page max

\section*{Overview}

TN Tech has nearly doubled its amount of externally-funded research from \dol[\mega\relax]{11} in 2015 to \dol[\mega\relax]{20} in 2020, and plans to reach \dol[\mega\relax]{40} of externally-funded research by 2025.

\section*{Intellectual Merit}

The acquisition of \textit{Warp~1} will enable research projects aligned with four of NSF's 10 Big Ideas.

\section*{Broader Impacts}

TN Tech has a strong history using HPC in workforce development at all levels.

\chapter{Project Description}

\section{Information about the Proposal}

\begin{description}[labelwidth=1.7in,leftmargin=\labelwidth,labelsep=0pt]
\item[Instrument Location:] Tennessee Tech University, Clement Hall Room NNN
\item[Instrument Type:] HPC cluster with high-speed networking and data storage for active jobs
\end{description}

\section{Research Activities to be Enabled}

This cluster will provide an easily-accessible computing facility that immediately benefits TN Tech's research efforts across several departments and research centers to advance both fundamental and applied research in science and engineering.

\subsection{Intellectual Merit}

The proposed \textit{Warp~1} cluster will enable research projects aligned with four of NSF's 10 Big Ideas
(Mike wanted some \textbf{\textit{bold italic}} text to verify, too).

\subsection{Users and Representative Scientific Research}

\begin{table}
\centering
\caption{\label{tab:users} Selection of externally-supported users impacted by the proposed equipment}
\begin{tabular}{P{1.7in}P{1.5in}P{1.6in}P{1.25in}} \toprule
{\bfseries Field} & {\bfseries User} & {\bfseries Applications} & {\bfseries Support} \\ \midrule
Department 1 & Faculty 1 & Materials science & NSF \\
             & Faculty 2 & Molecular dynamics & CWRU, DOD \\ \addlinespace

Department 2 & Faculty 3 & Fundamental science & NSF \\ \bottomrule
\end{tabular}
\end{table}

\newcommand{\mc}[2]{\multicolumn{#1}{c}{#2}} % Multicolumn, centered
\newcommand{\rb}[1]{\rotatebox{90}{#1}} % Rotated
\newcommand{\mcr}[2]{\mc{#1}{\rb{#2}}} % Multicolumn, centered, rotated
\newcommand{\mi}[1]{{\color{blue}\textbf{#1}}} % Multi-institutional
\newcommand{\ur}[1]{{\color{red}\textbf{#1}}} % Under-represented
\newcommand{\cmark}{\ding{52}}
\begin{table}[hbtp]
\caption{Summary of science drivers' communities, attributes, solutions from proposed upgrades}
\label{tab:science_drivers}
%\todo[inline]{Consider grouping power theft under smart grid, CFD and robotics under AI/ML}
\centering

\begin{tabular}{ll ccc l rr}
\toprule
& & \mc{3}{Needs} & & \mc{2}{People\footnotemark} \\
\cmidrule{3-5} \cmidrule{7-8}
\multicolumn{2}{l}{\textbf{Field of Study:} Driver} & \mcr{1}{CPUs} & \mcr{1}{GPUs} & \mcr{1}{Infiniband} & \mc{1}{Representative software used} & \multicolumn{1}{r}{\rotatebox{90}{Faculty}} & \multicolumn{1}{r}{\rotatebox{90}{Students}} \\
\midrule

\multicolumn{2}{l}{\textbf{Field 1:}} \\
&Project 1 & \cmark & \cmark & \cmark & Item 1, Item 2, Item 3 & 2      & \ur{3} \\
&Project 2 & \cmark & \cmark & \cmark & Item 4, Item 5, Item 6 & \mi{6} & \ur{3}  \\ 
&Project 3 & \cmark & \cmark & \cmark & Item 7, Item 8, Item 9 & 2      & \ur{10} \\
\addlinespace

\multicolumn{2}{l}{\textbf{Field 2:}} \\
&Project 4 &        & \cmark &        & Item 4, Item 10, Item 11 & 1      & 1 \\
&Project 5 & \cmark & \cmark & \cmark & Item 12, Item 13, Item 4 & \mi{3} & 6 \\
\bottomrule
\end{tabular}

\begin{minipage}{\textwidth}
\addtocounter{footnote}{-1}\footnotemark~\mi{Bold blue} faculty count indicates multi-institutional project,
\ur{bold red} student count indicates inclusion of under-represented groups.
\end{minipage}
\end{table}


This section details a selection of research projects.

\subsubsection{Next-Generation Genomics: Faculty 1, Department 1; Faculty 2, Department 2; Faculty 3, Department 3 (Other University)}

(5 faculty, 1 postdoc, 7 graduate students, 2 undergraduate students)
Funded projects in our labs use HPC resources to (1) generate genome-wide DNA data to inform on the management of endangered species \cite{torkamaneh2016}, (2) survey community composition based on DNA extracted from environmental sources \cite{watts2019}, and (3) investigate evolutionary processes in model organisms \cite{hurt2021}.

\section*{Broader Impacts}

Here we have a statement of broader impacts, which is often required as its own heading.

\subsection{Results from Prior NSF Support}

\subsubsection{Award N (\dol{123456}, 2/2019--1/2022) ``REU: Title'' (Lastname, Senior Personnel)}

\paragraph{Intellectual Merit}

Dr. Lastname is part of an expert group of faculty members who mentor REU participants in certain topics, by helping students to
\begin{enumerate*}[label=(\arabic*), itemjoin={{; }}, itemjoin*={{; and }}]
\item conceive, design, implement, and assess research projects in this area
\item learn diverse different toolsets in domains such as example 1, example 2, and example 3.
\end{enumerate*}

\paragraph{Broader Impacts}
Dr. Lastname's portion of the REU provided a research experience to 5 undergraduate students, including 2 female students.

\paragraph{Publications}
To date, this effort has resulted in 4 published papers \cite{paudel2019a,paudel2019b,paudel2018,mookiah2017}, with undergraduate and graduate students as first two authors.

\renewcommand{\bibname}{References Cited}
\printbibliography[heading=bibnumbered]

\end{document}
